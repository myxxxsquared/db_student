% !TEX program = xelatex

\documentclass[12pt]{ctexart}

\usepackage[colorlinks,linkcolor=red]{hyperref}

\title{NoSQL数据库的简单操作\\{\Large 大数据管理技术第一次实习作业}}
\author{张文杰 1500011394}

\begin{document}

\maketitle

\section{简介}

本次实习作业包含了Redis、MongoDB、Cassandra三种NoSQL数据库的安装配置和使用的练习。通过对于给定的学生数据的处理,比较了这三种不同的数据库在不同的增、删、改、查操作的易用性和效率差异,并运用这些差距分析了这三种NoSQL数据库的特点和适用场景。

报告中仅展示了简略的源代码与操作步骤,更为详细的内容已上传至Github \url{https://github.com/myxxxsquared/db_student}

\section{安装与配置}

我使用我的笔记本电脑进行本次的实习作业,操作系统为Ubuntu 18.04,主要使用 Miniconda(Python) 进行数据库操作和效率测试。
我选用了预编译版本的Redis、MongoDB、Cassandra,使用Ubuntu的包管理工具APT进行安装。搭建Miniconda环境,并安装数据库的Python驱动。
具体操作步骤见Github。

\section{NoSQL数据库的使用}

\subsection{Redis}

Redis数据库主要用于储存键值对和哈系表结构,在Python中使用只需调用相应的操作函数即可完成。Redis数据库没有额外的数据索引结构,因此对于数据库的访问只可以通过键名称进行。

Redis数据库中的增加操作可以直接使用hmset命令进行。通过键名称进行修改、查询、删除操作可以直接使用相应的命令hmset, hmget, del进行,然而如果想要通过数据值进行修改、查询、删除,则需要手动遍历整个数据库,筛选出需要操作的键名称,再进行相关操作。

\subsection{MongoDB}

MongoDB是用于储存文档的数据库,可以直接储存json文档。

MongoDB的增加使用的是insert\_many或insert\_one函数。修改使用的是update\_one或update\_many函数,其中的参数可以提供筛选信息和修改信息。删除使用的是delete\_one或delete\_many,提供了用于筛选删除的信息。查询使用的是find函数。

MongoDB提供了相当灵活的查询、修改指令,使得MongoDB的使用比Redis和Cassandra简单很多。MongoDB的查询可以直接使用一个json指定查询方式,例如``\{`schoolsup': `yes'\}''表示了查询`schoolsup'为`yes'的文档。MongoDB的修改操作也是使用一个json指定,例如``\{`\$set': \{`schoolsup': `yes'\}\}''表示将`schoolsup'设置为`yes'。

\subsection{Cassandra}

Cassandra数据库储存的是比较有结构化的数据,类似于一张二维表。不同于传统关系型数据库的是,Cassandra的储存结构为按列存储,并且之对于主键进行索引。因此,所有操作必须以主键为依据进行,这一点类似于Redis,这就使得对于其他的操作必须遍历整个数据库。

Cassandra采用了CQL语句操作,语法类似于SQL。使用SELECT, INSERT INTO, UPDATE, DELETE FROM进行查询、添加、更新、删除操作。与传统数据库不同之处是,CQL的操作只能以主键进行。为了达到以数据值操作必须遍历整个数据库。

\section{三种数据库的效率测试}

对于Redis、MongoDB、Cassandra三种数据库,我使用课程提供的student.csv数据,测试了以下11种操作的效率。

\begin{itemize}
    \itemsep0em
    \item \textbf{I\_all}批量插入
    \item \textbf{I\_one}逐个插入
    \item \textbf{D\_id}按照主键删除
    \item \textbf{D\_search}按照数据值删除,删除`schoolsup'为`yes'的记录
    \item \textbf{D\_all}全部删除
    \item \textbf{U\_id}按主键修改数据值,`reason'改为`other'
    \item \textbf{U\_search}按数据值修改数据值,修改`schoolsup'为`yes'的记录,`age'修改为15
    \item \textbf{U\_all}修改所有数据值,`age'修改为16
    \item \textbf{S\_id}按主键查询数据值,查询`famsize'
    \item \textbf{S\_search}按数据值查询数据值,查询`schoolsup'为`yes'的`famsize'
    \item \textbf{S\_all}数据库遍历,查询所有的`famsize'
\end{itemize}

测试的运行时间如表\ref{tab}所示

\begin{table}
    \centering
    \caption{三种数据库的测试运行时间\label{tab}}
    \begin{tabular}{l|c|c|c}
        \hline
        操作 & Redis & MongoDB & Cassandra\\
        \hline
        I\_all & - & \textbf{0.101084} & 0.917867\\
        I\_one & \textbf{0.331573} & 0.398549 & 1.094512\\
        D\_id & \textbf{0.062058} & 0.354701 & 0.737409\\
        D\_search & 0.083081 & \textbf{0.002577} & 0.110348\\
        D\_all & 0.013847 & \textbf{0.006087} & 0.107526\\
        U\_id & \textbf{0.064913} & 0.917833 & 0.800503\\
        U\_search & 0.074324 & \textbf{0.002887} & 0.121632\\
        U\_all & 0.082713 & \textbf{0.009799} & 0.859821\\
        S\_id & \textbf{0.076591} & 1.055133 & 1.722984\\
        S\_search & 0.076914 & \textbf{0.004754} & 0.021034\\
        S\_all & 0.082175 & 0.016752 & \textbf{0.022636}\\
        \hline
    \end{tabular}
\end{table}

\section{三种数据库效率差异的分析}

从表\ref{tab}中可以看出,对于以主键或键名为操作依据的插入、查询、修改、删除操作中,Redis数据库均表现出较高的效率,Cassandra次之,这正反映了两种数据库储存结构简单、专用于键名访问的特点。而对于按数据值的查询Redis和Cassandra需要遍历整个数据库,造成了较低的效率。

经过比较可以发现,Redis对于同行的访问效率高于Cassandra数据库,而Cassandra对于同列访问的效率高于Redis,这是由这两种数据库的不同存储结构造成的。Redis为按行存储的键值对,Cassandra为按列存储并且以主键为索引。

对于键进行按值操作中,MongoDB表现出较高的效率,这是由于只有MongoDB拥有除主键外的索引信息,提高了MongoDB的操作性能和数据库查询的灵活性。

经过以上比较,可以知道,Redis主要适用于按键名来访问的按行查询操作,MongoDB则适用于更加灵活的查询与操作较多的环境下,Cassandra则适用于对于按列的查询操作较多的环境。

\section{结论}

本次实习作业中通过对Redis、MongoDB、Cassandra三种数据库进行不同操作的比较,加深了对于这三种不同数据库功能和应用场景的理解。个人而言,我还是更喜欢MongoDB,因为它的操作简单,并且效率也比较高,写程序只需要几行代码就可以完成操作。不过另外两种数据库也适用于其他场合,遇到具体的问题,需要具体分析。

\end{document}
